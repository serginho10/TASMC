\section{Cómputo Móvil}

Sistema de computación en donde el usuario puede estar en movimiento, esto consiste en fabricar computadoras suficientemente pequeñas para ser fácilmente transportadas. Se tiene la necesidad de reemplazar los cables de conexión por una tecnología inalámbrica.

Este tipo de tecnología no solo representa una oportunidad de avance científico o computacional sino de implementar nuevas posibilidades de negocios como:

\begin{itemize}
	\item Aplicaciones financieras
	\item Gerencia de inventario
	\item Gerencias de servicios de campo
	\item Localización de productos
\end{itemize}

\subsection{Características de la Computación Móvil}

\begin{itemize}
	\item \textbf{Movilidad: }Implica la portabilidad basada en el hecho de que los usuarios llevan un dispositivo móvil a todas las partes a donde se dirigen, por lo tanto, los usuarios pueden iniciar el contacto en tiempo real con otros sistemas dondequiera que se encuentren.
	\item \textbf{Amplio alcance: }Es la característica que describe la accesibilidad de las personas, que se pueden localizar en cualquier momento.
	\item \textbf{Ubicuidad: }Se refiere al atributo de estar disponible en cualquier lugar en cualquier momento. Un terminal móvil en la forma de un teléfono inteligente o un PDA ofrece la ubicuidad.
	\item \textbf{Comodidad: }Es muy conveniente para los usuarios operar en el entorno inalámbrico, todo lo que necesitan es un dispositivo de Internet móvil, como un teléfono inteligente.
	\item \textbf{Conectividad Instantánea: }Los dispositivos móviles permiten a los usuarios conectarse de manera sencilla y rápida a la Internet e intranets, de otros dispositivos móviles y bases de datos.
	\item \textbf{Personalización: }Se refiere a la personalización de la información para los consumidores individuales.
    	\item \textbf{Localización de productos y servicios: }Conocer la ubicación física de los usuarios en cualquier momento es clave para ofrecer productos y servicios.
\end{itemize}