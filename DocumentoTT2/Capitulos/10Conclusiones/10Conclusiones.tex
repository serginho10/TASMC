\addcontentsline{toc}{chapter}{Conclusiones}
\chapter*{Conclusiones}

\lhead{\emph{Conclusiones}} % This is for the header on each page - perhaps a shortened title

A pesar de realizar distintas pruebas el rendimiento que obteníamos de la ubicación era de un porcentaje no tan razonable, ya que el API que se utiliza guarda dichas trayectorias en una nube, ente por el cual se debe ingresar usando una conexión a Internet, la terminal 1 cuenta con distintas redes repartidas en las salas de espera, las cuales pierden potencia cuando uno se encuentra dentro del pasillo principal (anden de pasajeros), de esta manera nos encontramos con distintas zonas muertas (lugares sin internet), mismas donde la respuesta a la ubicación en interiores iba a perder capacidad. Finalmente se realizaron mediciones de los valores de espectros magnéticos, los cuales son guardados en cada trayectoria trazada, de igual manera se encontró que dichos valores iban a ser variables dependiendo del flujo de personas, ya que ellos pueden acarrear distintos objetos que pueden generar variaciones al espectro magnético, de esta manera el rendimiento final que se pudo obtener utilizando esta API es del ochenta por ciento valor que fue calculado de manera propio. Es por esto que se plantea como trabajo a futuro poder montar una nueva infraestructura de sensores que permitan obtener una mejor respuesta. 

Además no se pudo obtener el apoyo del servicio externo que nos iba a permitir la alimentación de información real de servicios de hotelería y vuelos y como consecuencia se implemento un servicio el cual proporcionará esta información pero no con datos reales, de esta manera esos módulos quedan como un prototipo para obtener esa información a futuro emprendiendo negociaciones con distintos servicios que manejen este tipo de información.  

Finalmente se puede mencionar que el objetivo general de la aplicación queda cubierto ya que permite al usuario realizar una mejor gestión de su viaje mediante las distintas herramientas que proporciona la aplicación.