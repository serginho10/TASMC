% Capitulo 10

\chapter{Trabajo a Futuro} % Main chapter title

\label{TrabajoFuturo} % For referencing the chapter elsewhere, use \ref{Chapter1} 

\lhead{Capítulo 10 \emph{Trabajo a Futuro}} % This is for the header on each page - perhaps a shortened title

TASMC es una forma de organizar de manera adecuada el viaje del usuario, a pesar de que está aplicación tuvo que ser delimitada en  
cuanto a los objetivos que se plantearon al inicio, cabe resaltar que el objetivo principal quedó resuelto pues se desarrollo un sistema
que cumplirá con las espectativas del usuario en cuanto el desempeño y funcionalidad contenidas en el mismo, es necesario identificar aspectos 
que si bien no se contempla en el sistema hasta este momento, pueden integrarse como trabajo a futuro y extender las funcionalidades 
para brindar al usuario una mejor experiencia de viaje.

Dentro de las funcionalidades que pueden extender el alcance de la aplicación en una versión futura podemos identificar las 
siguientes: 

\begin{itemize}
 \item Ampliar la red de información agregando los aeropuertos de todo el mundo.
 \item Brindar el servicio de reservación dentro de la misma aplicación.
 \item Implementar una nueva solución para la localización en interiores, mediante una infraestructura externa que pueda ser 
 montable dentro de la edificación y de esta manera conseguir una mayor exactitud a la hora de localizar al usuario del aeropuerto.
 \item Añadir la información correspondiente de todos los servicios que brinda el aeropuerto.
\end{itemize}
