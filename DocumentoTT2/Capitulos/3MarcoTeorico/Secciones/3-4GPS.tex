\section{GPS}

\newcommand{\grad}{\hspace{-2mm}$\phantom{a}^{\circ}$}

GPS es la abreviatura de Global Positioning System ó Sistema de Posicionamiento Global en español. Es un sistema de radionavegación basado en satélites desarrollado y controlado por el Departamento de Defensa de Estados Unidos de América que permite a cualquier usuario saber su localización, velocidad y altura, las 24 horas del día, bajo cualquier condición atmosférica y en cualquier punto del globo terrestre. 

Después de la segunda guerra mundial, el Departamento de Defensa de Estados Unidos de América se empeñó en encontrar una solución para el problema del posicionamiento preciso y absoluto. Pasaron varios proyectos y experiencias durante los siguientes 25 años, incluyendo Loran, Transit, etc. Todos permitían determinar la posición pero eran limitados en precisión o funcionalidad. En el comienzo de la década de los 70, un nuevo proyecto fue propuesto, el GPS. 

El GPS tiene tres componentes: el espacial, el de control y el de usuario.

El componente espacial está constituido por una constelación de 24 satélites en órbita terrestre aproximadamente a 20200 km, distribuidos en 6 planos orbitales. Estos planos están separados entre sí por aproximadamente 60\grad   en longitud y tienen inclinaciones próximas a los 55\grad   en relación al plano ecuatorial terrestre. Fue concebido de manera que existan como mínimo 4 satélites visibles por encima del horizonte en cualquier punto de la superficie y en cualquier altura.

El componente de control está constituido por 5 estaciones de rastreo distribuidas a lo largo del globo y una estación de control principal (MCS- Master Control Station). Este componente rastrea los satélites, actualiza sus posiciones orbitales, calibra y sincroniza sus relojes. Otra función importante es determinar las órbitas de cada satélite y prever su trayectoria durante las 24 horas siguientes. Esta información es enviada a cada satélite para después ser transmitida por este, informando al receptor local donde es posible encontrar el satélite. 

El componente del usuario incluye todos aquellos que usan un receptor GPS para recibir y convertir la señal GPS en posición, velocidad y tiempo. Incluye además todos los elementos necesarios en este proceso, como las antenas y el software de procesamiento. 

\subsection{Funcionamiento GPS}

Los fundamentos básicos del GPS se basan en la determinación de la distancia entre un punto: el receptor, a otros de referencia: los satélites. Sabiendo la distancia que nos separa de 3 puntos podemos determinar nuestra posición relativa a esos mismos 3 puntos a través de la intersección de 3 circunferencias cuyos radios son las distancias medidas entre el receptor y los satélites. En la realidad, son necesarios como mínimo 4 satélites para determinar nuestra posición correctamente. 

Cada satélite transmite una señal que es recibida por el receptor, éste, por su parte mide el tiempo que las señales tardan a llegar hasta él. Multiplicando el tiempo medido por la velocidad de la señal (la velocidad de la luz), obtenemos la distancia receptor-satélite, (Distancia = Velocidad X Tiempo).

Sin embargo el posicionamiento satelital no es así de simple. Obtener la medición precisa de la distancia no es tarea fácil.

La distancia puede ser determinada a través de los códigos modulados en la onda enviada por el satélite  (códigos C/A y P), o por el análisis de la onda portadora. El receptor fue preparado de modo que solamente descifre esos códigos y ninguno más, de este modo él está inmune a interferencias generadas por fuentes naturales o intencionales. Esta es una de las razones para la complejidad de los códigos.  \cite{GPS}