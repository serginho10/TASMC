\section{Análisis de Riesgos}

El proceso de análisis de riesgos es de utilidad para conocer y de alguna manera tratar de reducir algunas actividades, ideas o actitudes que amenazan con la completa y eficaz elaboración del proyecto.

El primer paso para llevar a cabo el análisis de riesgos es identificarlos. Una manera sencilla para identificar los riesgos es incluirlos dentro de alguna de las siguientes clasificaciones: riesgos organizacionales, riesgos sobre el personal, riesgos tecnológicos, riesgos sobre cambios en los requerimientos y riesgos sobre las herramientas.

Una vez identificados y clasificados los riesgos se hace una valoración de la probabilidad de ocurrencia que tiene cada uno mediante una clasificación por rangos de probabilidad como la Tabla \ref{tab:probabilidadRiesgos}.

\begin{table}[h]
	\begin{center}
		\begin{tabular}{|c|c|}
			\hline \rowcolor[RGB]{51,153,255} 
				\textcolor{blanco}{\bf Probabilidad en \%} &
				\textcolor{blanco}{\bf Valoración} \\
			\hline 
				0\% - 10\% &
				Muy bajo \\
      		\hline \rowcolor[RGB]{240,248,255}
				10\% - 25\% &
				Bajo \\
			\hline 
				25\% - 50\% &
				Moderado \\ 
			\hline \rowcolor[RGB]{240,248,255}
				50\% - 75\% &
				Alto \\ 
			\hline
				75\% - 100\% &
				Muy Alto \\ 
			\hline
		\end{tabular}
	\end{center}
	\caption[Clasificación de Riesgos Conforme a su Probabilidad]{Clasificación de Riesgos Conforme a su Probabilidad} 
	\label{tab:probabilidadRiesgos}
\end{table}

Una vez que cada riesgo tiene asignada una valoración se procede a determinar el efecto que el riesgo tendrá en caso de que se llegara a cumplir. Una manera sencilla para determinar el efecto de cada riesgo es asignar una de las siguientes clasificaciones: catastrófico, serio, tolerable e insignificante. Cada una de las clasificaciones anteriores está ordenada en forma descendente conforme a la valoración de su impacto.

Cuando se han seguido todos los pasos para el proceso de análisis de riesgos el resultado es una tabla donde se muestra el nombre del riesgo, su clasificación, su valoración y su impacto. Los riesgos presentados en la Tabla \ref{tab:analisisRiesgos} deben ser ordenados de manera descendente conforme a su impacto.

\begin{table}[h]
	\begin{center}
		\begin{tabular}{|>{\columncolor[RGB]{51,153,255}}p{6.6cm}|c|c|c|}
			\hline \rowcolor[RGB]{51,153,255} 
				\textcolor{blanco}{\bf Riesgo} &
				\textcolor{blanco}{\bf Clasificación} &
				\textcolor{blanco}{\bf Valoración} &
				\textcolor{blanco}{\bf Efecto} \\		
			\hline
				\textcolor{blanco}{\bf Mala comunicación entre los integrantes del equipo} &
				Organizacional &
				Alto &
				Serio \\
			\hline
				\textcolor{blanco}{\bf Recursos insuficientes para concluir el proyecto} &
				\cellcolor[RGB]{240,248,255}  Organizacional &
				\cellcolor[RGB]{240,248,255}Alto &
				\cellcolor[RGB]{240,248,255}Serio \\
      		\hline 
				\textcolor{blanco}{\bf Retraso de las actividades del proyecto} &
				Organizacional &
				Bajo &
				Serio \\
			\hline  
				\textcolor{blanco}{\bf Falta de responsabilidad de los integrantes del equipo} &
				\cellcolor[RGB]{240,248,255}Personal &
				\cellcolor[RGB]{240,248,255}Bajo &
				\cellcolor[RGB]{240,248,255}Serio \\
			\hline
				\textcolor{blanco}{\bf Mala distribución de actividades} &
				Organizacional &
				Bajo &
				Serio \\
			\hline 
				\textcolor{blanco}{\bf Permiso denegado para realizar trabajos dentro del AICM} &
				\cellcolor[RGB]{240,248,255}Tecnológico &
				\cellcolor[RGB]{240,248,255}Moderado &
				\cellcolor[RGB]{240,248,255}Serio \\
			\hline
				\textcolor{blanco}{\bf Mal control de las versiones del proyecto.} &
				Organizacional &
				Moderado &
				Serio \\
			\hline 
				\textcolor{blanco}{\bf Baja definitiva de alguno de los integrantes del equipo.} &
				\cellcolor[RGB]{240,248,255}Personal &
				\cellcolor[RGB]{240,248,255}Muy Bajo &
				\cellcolor[RGB]{240,248,255}Serio \\
			\hline
				\textcolor{blanco}{\bf Cambios en los requerimientos del sistema por parte de los sinodales} &
				Requerimientos &
				Alto &
				Serio \\
			\hline 
				\textcolor{blanco}{\bf Enfermedad de alguno de los miembros del equipo.} &
				\cellcolor[RGB]{240,248,255}Personal &
				\cellcolor[RGB]{240,248,255}Bajo &
				\cellcolor[RGB]{240,248,255}Tolerable \\
			\hline
				\textcolor{blanco}{\bf Rendimiento no competitivo del sistema} &
				Tecnológico &
				Alto &
				Serio \\
			\hline 
				\textcolor{blanco}{\bf Ausencia de algún integrante del equipo por un periodo prolongado de tiempo} &
				\cellcolor[RGB]{240,248,255}Personal &
				\cellcolor[RGB]{240,248,255}Muy Alto &
				\cellcolor[RGB]{240,248,255}Tolerable \\
			\hline
				\textcolor{blanco}{\bf Falta de dominio de las herramientas de desarrollo} &
				Personal &
				Moderado &
				Tolerable \\
			\hline 
				\textcolor{blanco}{\bf Falla en los dispositivos móviles de prueba} &
				\cellcolor[RGB]{240,248,255}Tecnológico &
				\cellcolor[RGB]{240,248,255}Alto &
				\cellcolor[RGB]{240,248,255}Tolerable \\
			\hline
				\textcolor{blanco}{\bf Funcionamiento inadecuado en la implementación de alguna tecnología después de haber sido calificada como adecuada en el Estudio de Factibilidad} &
				Tecnológico &
				Alto &
				Serio \\
			\hline 
				\textcolor{blanco}{\bf Incorrecta definición de la problemática del proyecto} &
				\cellcolor[RGB]{240,248,255}Organizacional &
				\cellcolor[RGB]{240,248,255}Muy Alto &
				\cellcolor[RGB]{240,248,255}Serio \\
			\hline
				\textcolor{blanco}{\bf Acceso negado a los servicios web de Amadeus.} &
				Recursos &
				Moderado &
				Serio \\
			\hline
		\end{tabular}
	\end{center}
	\caption[Análisis de Riesgos]{Análisis de Riesgos} 
	\label{tab:analisisRiesgos}
\end{table}

\textbf{Identificación de Riesgos}
\newline
Los riesgos han sido identificados clasificados, según el foco de interés, de la siguiente manera:

\begin{itemize}
	\item Ambiente del proyecto
	\item Inventario de activos e intangibles
	\item Mantenimiento de software y hardware
	\item Equipo de trabajo 
	\item Errores de estimación de costos 
\end{itemize}
\clearpage
\textbf{Estrategias para la mitigación de riesgos}
\newline

\begin{table}[h]
	\begin{center}
		\begin{tabular}{|p{14.2cm}|}
			\hline \rowcolor[RGB]{51,153,255} 
				\textcolor{blanco}{\bf Estrategia de mitigación} \\ 
			\hline 
				Seguimiento semanal de los hitos definidos por cada actividad en el desarrollo del proyecto. \\
      		\hline \rowcolor[RGB]{240,248,255} 
				Continúa capacitación del personal durante las horas de trabajo. \\
			\hline 
				Designación de considerable tiempo para esta actividad. \\ 
			\hline \rowcolor[RGB]{240,248,255}
				Conocimiento y seguimiento de la situación laboral y personal de cada miembro del equipo. \\ 
			\hline 
				Realización de pruebas como actividades de capacitación de las tecnologías a usar antes de ser implementadas. \\ 
			\hline \rowcolor[RGB]{240,248,255}
				Realización de pruebas al sistema para poder evitar el mal funcionamiento de alguno de los módulos \\ 
			\hline
				Planificación de todas las actividades antes de comenzar con el proyecto y fijar una fecha de entrega anterior a la acordada con el cliente pues esto nos dará una holgura de tiempo \\ 
			\hline \rowcolor[RGB]{240,248,255}
				Buscar y utilizar herramientas que se adapten a todos los dispositivos \\ 
			\hline
				Mantener una actitud positiva para manejar  los problemas y regirse conforme a la planeación \\ 
			\hline \rowcolor[RGB]{240,248,255}
				En caso de recibir una respuesta negativa para el acceso a los servicios web de Amadeus, fijar una fecha para desarrollar un servicio web con la información necesaria para implementar la funcionalidad del sistema que se vería afectada.  \\ 
			\hline
		\end{tabular}
	\end{center}
	\caption[Plan de Mitigación]{Plan de Mitigación} 
	\label{tab:planMitigacion}
\end{table}

\textbf{Planes de contingencia}
\newline

\begin{table}[h]
	\begin{center}
		\begin{tabular}{|p{14.2cm}|}
			\hline \rowcolor[RGB]{51,153,255} 
				\textcolor{blanco}{\bf Planes de contingencia} \\
			\hline 
				Compensar el tiempo de retraso con horas de trabajo extra. \\
      		\hline \rowcolor[RGB]{240,248,255}
				Capacitación personal para el personal incompetente en horas extra de trabajo. \\
			\hline 
				Dialogo en el  equipo de trabajo si existe una mala definición en la problemática del proyecto para hacer un replanteamiento de la problemática. \\ 
			\hline \rowcolor[RGB]{240,248,255}
				Distribución del trabajo entre los miembros del equipo aumentando jornadas de trabajo. \\ 
			\hline 
				Elección de una tecnología secundaria compatible con el proyecto. \\ 
			\hline \rowcolor[RGB]{240,248,255}
				Ocupar la holgura de tiempo y trabajar más del tiempo estipulado. \\ 
			\hline
				Reparar el modulo o crear uno nuevo para evitar contratiempos. \\ 
			\hline
		\end{tabular}
	\end{center}
	\caption[Plan de Contingencia]{Plan de Contingencia} 
	\label{tab:planContingencia}
\end{table}

\begin{center}
	{\bf \small Nota: Todos los riesgos antes mencionados pueden presentarse en cualquier parte}
	{\bf \small del proyecto por eso  los planes de mitigación y/o contingencia no han sido fechados.}
\end{center} 