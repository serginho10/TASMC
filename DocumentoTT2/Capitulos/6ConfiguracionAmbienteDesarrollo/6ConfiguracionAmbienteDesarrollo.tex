% Capitulo 6

\chapter{Configuración del Ambiente de Desarrollo} % Main chapter title

\label{ConfiguracionAmbienteDesarrollo} % For referencing the chapter elsewhere, use \ref{Chapter1} 

\lhead{Capítulo 6 \emph{Configuración del Ambiente de Desarrollo}} % This is for the header on each page - perhaps a shortened title

Esta tarea consiste en configurar los ambientes tanto físicos como técnicos para el proyecto. Esta tarea involucra a los desarrolladores de software en el ambiente técnico de desarrollo, además se realizan pruebas de concepto sin necesariamente implementar un requerimiento.

\section{Configuración Web Service}

\begin{itemize}
	\item \textbf{Nombre aplicación: } TASMC.
	\item \textbf{Tipo de Proyecto: } Aplicación PHP.
\end{itemize}

\section{Configuración Aplicación Móvil }

\begin{itemize}
	\item \textbf{Nombre aplicación: } TASMC.
	\item \textbf{Tipo de Proyecto: } Blank Activity.
	\item \textbf{SDK Mínimo Requerido: } API 19 – Android 4.4 (KitKat).
	\item \textbf{Configuraciones: } habilitado Automatic Reference Counting (ARC), aplicación universal.
\end{itemize}

\section{Configuración Base de Datos SQLite}

\begin{itemize}
	\item \textbf{Nombre base de datos: } tasmc.
	\item \textbf{Importaciones: }
	\begin{itemize}
		\item android.database.Cursor para recuperación de datos.
		\item android.database.sqlite.SQLiteDatabase para manejo de SQLite dentro de la aplicación.
	\end{itemize}
\end{itemize}

\section{Configuración Google Maps API}

\begin{itemize}
	\item Instalar el SDK de Android. 
	\item Descargar y configurar el SDK Google Play services, que incluye la API de Google Maps para Android. 
	\item Obtener una clave de API. Dar de alta el proyecto en la consola de las API de Google, y obtener un certificado de firma para la aplicación. 
	\item Añadir los ajustes necesarios en el manifiesto de la aplicación. 
	\item Añadir un mapa de la aplicación. 
\end{itemize}

\section{Configuración IndoorAtlas API}

\begin{itemize}
	\item Obtener una clave de API y la contraseña.
	\item Dar de alta una edificación, niveles y plantas.
	\item Añadir los IDs correspondientes a la edificación y  sus plantas.
\end{itemize}

\section{Configuración IDE}

\begin{itemize}
	\item Obtención de herramientas del SDK de Android
	\begin{itemize}
		\item Android SDK Tools
		\item Android SDK Plataform-tools
		\item Android SDK Build-tools
	\end{itemize}
	\item Obtención de API 19 (Android 4.4)
	\begin{itemize}
		\item SDK Plataform
		\item Samples for SDK
		\item ARM EABI v7 a System Image
		\item Intel x86 Atom System Image
		\item Google APIs (x86 System Image)	
		\item Google APIs (ARM Sytem Image)
		\item Glass Development Kit Preview
		\item Sources for Android SDK
	\end{itemize}
	\item Obtención de Google Services
	\begin{itemize}
		\item Google Play Services
		\item Google Repository
		\item Google USB Driver
		\item Google Web Driver
		\item Google Repository
		\item Google Play Licensing Library
	\end{itemize}
\end{itemize}