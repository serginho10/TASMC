\section{Navegación y Gestos}
Como ya se explicó anteriormente la aplicación móvil consta de ocho vistas principales, para poder visualizar cada una de estas, 
es necesario abrir el menú principal de la aplicación deslizando el dedo de izquierda a derecha o presionando el botón en la 
esquina superior izquierda. Para lograr esto se ha utilizado el patrón de diseño conocido como "Navigation Drawer" y haciendo uso
del componente Actionbar para obtener una fácil navegación asi como utilizar los patrones del material de diseño de Google para 
ofrecer un diseño agradable al usuario.

La Navegación dentro de la aplicación móvil esta basada en cinco gestos sobre la pantalla del dispositivo móvil, a continuación 
se describe cada uno de estos y se menciona en que vistas se utilizan y cuales son las operaciones que nos permiten realizar.

\subsection{Clic Simple}
Este gesto es el más simple y utilizado dentro de la aplicación, consiste únicamente en seleccionar la opción a la que se desea 
acceder mediante un clic simple sobre dicha opción o botón. Posterior a esto se realizara la opción correspondiente a la opción 
seleccionada. Dentro de la aplicación, este gesto es utilizado en: 

\begin{itemize}
 \item Acceso a los distintos módulos dentro del menú principal de TASMC.
 \item Selección de los distintos valores en los formularios de búsqueda de Hoteles y Vuelos, así como el envío de datos por medio 
 de su respectivo botón de búsqueda.
 \item Clic sobre el botón de acción para el despliegue de filtros de Hoteles y Vuelos.
 \item Clic sobre el botón de acción para hacer llamado a una nueva actividad como es la generación de un nuevo itinerario o equipaje.
 \item Clic sobre botones para mostrar la información referente a mapa y servicios dentro del AICM.
 \item Acceso a la edición de listas de equipaje e itinerarios de viaje.
 \item Puntualizar origen y destino mediante marcadores para generar la ruta al AICM.
\end{itemize}

\subsection{Desplazar}
Este gesto consiste en desplazarse por la pantalla de arriba hacia abajo y viceversa. 
Dejando pulsada la pantalla movemos el dedo arriba o abajo. Este gesto es utilizado principalmente en las vistas que contienen 
listas como son:

\begin{itemize}
 \item La lista del menú principal.
 \item Lista de Hoteles Disponibles.
 \item Lista de Equipaje y objetos.
 \item Lista de Itinerarios de Viaje.
 \item Lista de Llegadas y Salidas Nacionales e Internacionales.
\end{itemize}

\subsection{Deslizar}
Este gesto consiste en mantener oprimida la pantalla del dispositivo y deslizar el dedo a lo ancho de la pantalla para intercambiar 
la vista que se esté presentando. Su uso primordial es:

\begin{itemize}
 \item Intercambiar las vistas para la búsqueda de vuelos; ida y redondo.
 \item Intercambio de mapas de Planta Alta y Planta Baja.
 \item Navegación entre las vistas de Salidas y Llegadas.
 \item Deslizar el menú principal.
\end{itemize}

\subsection{Pellizcar}
Es la típica acción que hacemos juntando o separando dos dedos para, por ejemplo, hacer zoom en una imagen. Este gesto es utilizado en:
\begin{itemize}
 \item El zoom de mapas del AICM.
 \item El zoom del mapa de la ruta hacia el AICM.
\end{itemize}

\subsection{Arrastrar}
Por último, este gesto consiste en mantener oprimida el área de selección y deslizar el dedo hacia arriba o hacia un lado para tener 
acceso a la información o funcionalidades correspondientes. 
\begin{itemize}
 \item Este gesto se utiliza en las vistas de Hoteles y Vuelos Disponibles para actualizar la información.
 \item Es aplicado en la lista de itinerarios, dejando presionado un itinerario y desplazandoló hacia la derecha el itinerario es 
 eliminado de la base de datos.
\end{itemize}
